\section{Linear Programming}


\subsection{Linear Programming}

\paragraph{Goals}
The goal is to minimise a \textit{objective} function that is subject to a set of constraints.

\paragraph{Standard Form of Objective Function}
Typically the objective function is given as \[
  \sum_{j = 1}^{n} c_j x_j
.\] 
Without loss of generality, we can always treat this as a maximisation problem,
by reversing the sign of the objective function when the problem requires minimisation.

We assume constraints are of form \[
  sum_{j = 1}^n a_{ij} x_{j} \leq b_i
\]  where \(1 \leq i \leq m\), \(1 \leq j \leq n\) and \(x_j \geq 0\).

\paragraph{Compaction as Vectors}
Let \(\vec{x}\) represent a column vector \[
  x = \begin{pmatrix} x_1 \\ \vdots \\ x_n \end{pmatrix} 
.\] 

We can define a partial ordering on vectors in \(\mathbb{R}^n\) where \(\vec{x} \leq \vec{y}\) if and only if,
for \[
  x_j \leq y_y \quad \forall 1 \leq j \leq n
.\] 

Then the problem can be written as a maximisation of \(\vec{c}^T \vec{x}\) subject to the
(matrix-vector) constraints
\begin{align*}
  A\vec{x} &\leq b\\
  \vec{x} &\geq 0.
.\end{align*}

\paragraph{Complexities of Solutions}
Linear programming can be solved in polynomial time. However we usually use a derivative of the
SIMPLEX algorithm, which has an expontial worst case but, a very efficient-
\textit{average} case..

\paragraph{Dealing with Equality Constraints in the Standard Form}


\subsection{Food Example}
\paragraph{Problem Context}
Foods \(F1, \ldots, F_n\). For each \(F_i\), there exists
 \begin{itemize}
   \item Price per gram \(p_i\) 
   \item Calories per gram \(c_i\)
   \item \(v_{i, j}\) which is the milligrams of vitamin
     \(V_j, j = 1, \ldots 13\), in one gram of \(F_i\) 
\end{itemize}

\paragraph{Task}
Find a combination of foods such that
\begin{itemize}
  \item Total calories is \(2000\)
  \item The daily totak intake of all \(V_j\) of  \(w_j\) is met
\end{itemize}



% \subsection{Formulating linear programs}
% \subsection{Linear programming and integer linear programming}
