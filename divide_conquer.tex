
\section{Divide-and-Conquer Method}

% \subsection{Asymptotic behavior, recurrences, summations, estimations}
% \subsection{Weighing coins}
\subsection{Counting Inversions}

\paragraph{Brute Force}
Brute force will take \(O(n^2)\) time. Any method of looking at entries
one by one will have have at worst, a quadratic time complexity
since there can be \(\frac{n^2}{2}\) inversions.

\paragraph{Divide and Conquer Method}
We can split the array into two equal parts \(A_{\hi}\) and \(A_{\lo}.\)

Then, for all of the elements in \(A_{\hi}\) and the total inversions is
the number of elements in \(A_{\lo}\) that are less than the elements of
\(A_{\hi}\).

We may sort both \(A_{\hi}\) and \(A_{\lo}\).
Then, we seek to merge the arrays together. Every time we pull an element
from \(A\), we add the number of elements remaining in \(A_{\lo}\)
to the total number of inversions as, all of those elements are greater than
our current element from \(A_{\hi}\) but to the left of our current element.

\paragraph{Time Complexity of Divide and Conquer Method}
The divide and conquer method will the same time complexity as
merge sort and thus runs in \(\Theta (n)\).

\subsection{The Master Theorem}


\paragraph{Setup Master Theorem}
Let \(a \geq 1\)  be and integer and \(b > 1\) be a real number,
\(f(n) > 0\) be a non-decreasing function defined on the positive integers.
Then, \(T(n)\) is the solution of the recurrence
\[
    T(n) = a T\left( \frac{n}{b} \right) + f(n).
\]

Then, we define the critical exponent \(C^* = \log_b(a)\) 
and the critical polynomial \(n^{c^*}\).

\paragraph{Master Theorem}
\begin{enumerate}
    \item If \(f(n) = O(n^{c^*})\) for some \(\epsilon > 0\) then,
    \(T(n) = \Theta(n^{c^*})\).
    \item If \(f(n) = \Theta(n^{c^*})\) then,
    \(T(n) = \Theta (n^{c^*})\log n\).
    \item  If \(f(n) = \Omega(n^{c^* + \epsilon})\) for some \(\epsilon > 0\)
    and, for some \(c < 1\), and some \(n_0\),
    \[a f(\frac{n}{b}) \leq c f(n)\]
    holds for all \(n > n_0\) then, \(T(n) = \Theta(f(n))\).
\end{enumerate}

If the conditions above do not hold then, the master theorem is not
applicable.


% \subsection{Applications of the master Theorem: median of medians}