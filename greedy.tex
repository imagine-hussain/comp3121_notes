
\section{The Greedy Method}

\subsection{When greed pays off - foundations of the Greedy Method}

\paragraph{What is a Greedy Problem}
A greedy algorithm will divide a problem into stages and rather than exhaustively
searching through all combinations of options in all stages, it only
considers the choice that is the best for the current stage.

The idea is that the search space is reduced however, it is not
necessarily given that the solution is found using a greedy point.

\paragraph{Proofs of Greedy Algorithm}
There are two main methods of proof.
\begin{enumerate}
    \item \textbf{Greedy Stays Ahead:} This proves that at every stage,
    no other algorithm can do better than the proposed algorithm.
    \item \textbf{Exchange Argument:} Consider an optimal solution and
    gradually transform it to the solution found by the proposed algorithm
    without making it any worse.
\end{enumerate}
These methods are analogous to proof by induction and contradiction
respectively.

\subsection{Activity Selection problem}
\paragraph{Problem Statement}
There is a list of \(n\) activities with starting times \(s_i\).
and finishing time \(f_i\). Schedule the activities such that no two
activities overlap. Maximise for the total number of activities.

\paragraph{Solution}
Among the activities that do not conflict with the previously chosen
activities, choose the activity with the earliest end-time.
Ties may be broken arbitrarily.


% \subsection{Discrete (0–1) Knapsack Problem}
% \subsection{File compression: Huffman Codes}
% \subsection{Directed acyclic graphs and topological sorting}
% \subsection{Dijkstra's algorithm}
% \subsection{Minimum spanning trees}