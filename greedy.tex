
\section{The Greedy Method}

\subsection{When greed pays off - foundations of the Greedy Method}

\paragraph{What is a Greedy Problem}
A greedy algorithm will divide a problem into stages and rather than exhaustively
searching through all combinations of options in all stages, it only
considers the choice that is the best for the current stage.

The idea is that the search space is reduced however, it is not
necessarily given that the solution is found using a greedy point.

\paragraph{Proofs of Greedy Algorithm}
There are two main methods of proof.
\begin{enumerate}
    \item \textbf{Greedy Stays Ahead:} This proves that at every stage,
    no other algorithm can do better than the proposed algorithm.
    \item \textbf{Exchange Argument:} Consider an optimal solution and
    gradually transform it to the solution found by the proposed algorithm
    without making it any worse.
\end{enumerate}
These methods are analogous to proof by induction and contradiction
respectively.

\subsection{Activity Selection problem}
\paragraph{Problem Statement}
There is a list of \(n\) activities with starting times \(s_i\).
and finishing time \(f_i\). Schedule the activities such that no two
activities overlap. Maximise for the total number of activities.

\paragraph{Solution}
Among the activities that do not conflict with the previously chosen
activities, choose the activity with the earliest end-time.
Ties may be broken arbitrarily.

\paragraph{Proof}
Correctness is proved with the \textit{exchange argument} to show that
any optimal solution can be transformed into our greedy solution.
\begin{enumerate}
    \item Find the first place at which the optimal solution violates
    the greedy choice.
    \item Replace the activity chose with the greedy choice.
    Clearly, the number of activities is the same.
    Also, we know there are no conflicts that have been created because,
    we are working left-to-right and have chosen the first conflict.
    So, no conflict happens before the start of the greedy's choice.
    Also, the finish time of greedy is no greater than the finish
    time of the optimal solution. As such, there is no conflict
    on the right-side of the greedy choice either.
\end{enumerate}

\paragraph{Complexity}
We can sort using the finishing time as the key in \(n\log n\).
Then, loop through all the activities linearly for a total
time of \(O(n \log n)\).


% \subsection{Discrete (0–1) Knapsack Problem}
% \subsection{File compression: Huffman Codes}
% \subsection{Directed acyclic graphs and topological sorting}
% \subsection{Dijkstra's algorithm}
% \subsection{Minimum spanning trees}

% \subsection{Cell Towers}

% \paragraph{Problem}

% \paragraph{}

\subsection{Job Lateness}
\paragraph{Problem}
At a start time \(T_0\) and a list of \(n\) jobs with duration times
\(t_i\) and deadlines \(t_i\).

Assume that only one job can be done at a time and all jobs need to
be completed

\paragraph{Solution}
Ignore the duration. Then, choose jobs in terms of ascending deadlines.

\paragraph{Proof}
Consider an optimal solution.
We say \(i, j\) is an inversion if \(i\) is scheduled before
\(j\) but, \(j\) is scheduled after \(i\).

\subsection{Huffman Codes}
\paragraph{Array Merging Problem}
Are given \(n\) sorted arrays of different
sizes. May only merge \(2\) at a time.

\paragraph{Huffman Code}
Given a set of symbols, encode these symbols into a binary string
that can be decoded unambiguously.

\paragraph{Naive Solution: Ascii}
Designate each symbol as a character and assign a unique integer
to that character. Given \(n\) characters, you require
\(\lceil n \log n \rceil\)
bits.
Wastes lots of space.

\subsection{Tsunami Warning}

\paragraph{Situation}
There are \(n\) radio towers to broadcast tsunami warnings.
You are given the \((x, y)\) coordinates of each tower and its
radius of range. When a tower is activated, all towers within that radius
will also activate, causing other towers to activate andso on.

\paragraph{Task}
Design and algorithm to ind the fewest number of towers needed to be equipped

\paragraph{Attempt 1 - Bad}
Find activated tower with the largest radius. Place a sensor at this tower.
Then find and remoe all activated towers. Repeat.
\paragraph{Attempt 2 - Bad}
Find unactivated tower with largest number of towers in range. If none, place
a sensor at the left-most tower. Repeat.

\paragraph{Motivation}
Consider the towers as nodes on a directed graph where an edge \(a_{ij}\)
implies \(i\) causes an activation of \(j\). Observe that the relation is
symmetric.

Suppose that a tower \(a\) causes \(b\) to be activated and vice-versa.
Then, we never want to activate \(a\) and \(b\). Consequently,
we never want to activate more than two elements in a cycle.

Let \(S\) be a subset of towers such that activating any tower in \(S\)
causes the activation of all towers in \(S\). As such, we may treat \(S\) 
as a singular unit; a super-tower.

\paragraph{Strongly Connected Components}
Given a directed graph \(G = (V, E)\) and a vertex \(v\), the strogly connected
components of \(G\) containing \(v\) consists of all vertices
\(u \in  V\) such that there is a path in \(G\) from \(v\) to \(u\) and a path
from \(u\) to \(v\). We denote this as \(C_v\).
In terms of our problem, strongly connected components are maximal
super-towers.

\paragraph{Finding Strongly Connected Components}
Given a graph \(G = V, E\), create \(G' = (V, E')\) so that
\(E'\) has all edges reversed from \(E\).)

We claim that \(u \in C_v\) iff and only if \(u\) is reachable from \(v\)
and \(v\) is reachible from \(u\). Equivalently, \(u\) is reachable from
\(v\) in both \(G\) and \(G'\).

\paragraph{All Strongly Connected Components Time Complexity}
Observe that this may require a BFS for each component, which runs in
\(O(V + E)\). Doing so for all \(V\) vertices leads to a total of
 \(O(V(V+E))\) time.

\paragraph{Kosaraku's Algorithm}
Kosaraju's and Tarjan's algorithm can find all strongly connected components
in linear time as \(O(V+E)\). See this in CLRS \(\textsection 22.5\).

\paragraph{Consensation Graph}
Clearly each \(v \in  V\) exists in only one stronlgly connected component.
That is, the strongly connected components form a partition of \(V\).
Let \(C_G\) be the condensation graph of \(G\). That is, \[
    \sum_G = (C_G, E^*)
\] where,
\[
    E^* = \{
        (C_{u_1}, C_{u_2}) : (u_1, u_2) \in  E, C_{u_1} \neq C_{u_2}.
    \} 
.\]
The vertices of \(\sum_G\) are strongly connected components of \(G\).
The edges of \(\sum_G\) correspend to those edges of \(G\) that are not within
a strongly connected component with duplicated ignored.

\paragraph{Returning To Tsunami Problem with Condensation Graphs}
\begin{enumerate}
    \item Find the condensation graph.
        This gives a set of \textit{super tower}, for which we know what
        others get activated. Observer that we want our super-towers to
        be maximal.
    \item Now need to decide what super towers to put towers on.
        Observe that the condensation graph is a directed, acyclic and
        anti-symmetric graph.
\end{enumerate}

\paragraph{Topological Sorting}
Let \(G = (V, E)\) be a directed graph with \(n = |V|\).
A topological sort of \(G\) is a linear ordering of its vertices
\(\sigma: V \to  \{1, \ldots, n\} \) such that if there is an edge
\(v, w \in  E\) then \(v\) preceded \( w \) in the order. That is,
\( \sigma(v) < \sigma(w)\).

Observe that this may only occur on an acylic graph, hence the need for
the condensation graph. Also, the topological sort may not be unique.

This algorithm may be done in \(O\left( V + E \right) \).
Recall that linear algorithms like this are equivalent to reading the graph. 
As such, they may be done \textit{for free}.

\subparagraph{Djikstra - Shortest Single Path}

\paragraph{Algorithm Outline}
Create a boolean array \(S\) with each index \(i\) covering if \(i\)
has had its shortest path weight found.

For each vertex \(v\) maintain a \(d_v\) equal to the weight of the
shortest know path \(s \to v\).
Initially \(d_s = 0\) and \(d_v = \infty\).

\paragraph{Correctness}
\subparagraph{Claim} Suppose that \(v\)is the next vextex to be added to \(S\)
then \(d_v\) \(d_v\) is teh lenght of the shortest path from \(s\) to \(v\).

TODO: Finish this off

\paragraph{Notation for following}
Let \(n = |V|, m = |E|\). Let the vertices be labelled \(1, \ldots, n\)
with source \(n\).

\subparagraph{Array Solution}
Store \(d_i\) in array of length \(n\). At each stage
\begin{enumerate}
    \item Perfor linear search of \(d\) ignoring vertices already in \(S\).
        Select vertext \(v\) with smallest \(d[v]\) to be added to \(S\).
    \item For each outgoing edge from \(v\) to some \(z \in V \backslash S\),
        update \(d[z]\) if necessary.]
\end{enumerate}

At each \(n\) steps, we perform a linear scan on array of length \(n\)
and also update \(d\) in constant time at most once for each edge.

Thus, the complexity is \(O(n^2 + m)\).) IN a simply graph, this can be
simiplified to \(O(n^2)\).
This solution is okay for dense graphs but struggles when it comes to
sparse graphs.

\paragraph{Motivation for Better Structure}
We need to support
\begin{enumerate}
    \item Finding \(v\in V \backslash S\) with smallest \(d_v\)
    \item For each outdoing edge \(w, z\) update \(d_z\) if needed.
\end{enumerate}
The first is a a slow linear search. However we can skip over vertices
already in \(S\). This can be implemented by deleting theo\(d_v\)
from the data structure all together.

For this, we may use a min-heap priority queue as it has fast deletion
and access of the minimum element. However, the standard heap doesn't
allow updating values.

\paragraph{Augmented Heap}
The heap may be represented as an array of size \(n\) wwhere the left child
of \(A[j]\) is stored in \(A[2j]\) and the right child in \(A[2j + 1]\).

Changing \(d[i]\) is now \(O\log n\). It requires a lookup of ]
TODO::w

\subsection{Minimum Spanning Trees}

\paragraph{Definition}
A minimum spanning tree (mst) \(T\) of a connexted graph is a sub-graph
of \(G\) which is a tree that contains all vertices of \(G\) and minimises
the sum of all edges.

\paragraph{Lemma}
Let \(S\) be a non-empty proper subsert of all vertices of \(G\).
Assume that \(e = (u, v)\) is an ege such that \(u \in S\)
and \(V \not\in S\) and \(e\) is of minimal length of all edges with this
property. Then, \(e\) must belong in all msts of \(G\).

\paragraph{Kruskal's Algorithm}
Sort all the edges \(E\) in increasing order by weight.
Then, starting from lowest weight to highest, if adding an edge will
not result in a cycle, then add it to the graph. Otherwise, discard the edge
and return it.

\paragraph{Uniqueness of Kruskal}
Consider where all weights are distinctive. Consider edge \(e = (u, v)\)
add during Kruskal's algorithmn and let \(F\) be the forest in its state before
adding \(e\)..

let \(S\) be the set of vertices reachable from \(u \in F\). Then clearly
\(u \in S, v\not\in S\).

The original grpah does not contain any edges shorter than \(e\) with one ending
in \(S\) and the other not in \(S\). If it existed, then it would have
already been added in during and earlier iteration of the algorithm.

\paragraph{Implementing}
The sorting is a solved problem. However, finding cycles is more difficult.
For this, we use a union-find.

\paragraph{Union-find}
The unino-find handles disjoint sets and has atleast the following operations.
\begin{enumerate}
    \item Creation of the set returns a sstructure where all vertices are placed into distinct singleton sets.
    \item Find: Returns the label of the set to which the given element belongs. Returns in \(O(1)\)
    \item Union: Takes two sets and merges them toghether into the union of the sets.
        Takes linear time but, by amortized analysis we see that the first \(k\) operations run in a total of \(O(k\log k\).
        Note that this it NOT equivalent to an average case analysis but rather, the worst-case of an aggregate of operations.
\end{enumerate}

\paragraph{Additional Data Structures}
The simiplest union-find requires
\begin{enumerate}
    \item An array \(A\) such that \(A[i] = j\) means that \(i\) belongs
        to the set with represtative \(j\).]
    \item Array \(B\) where \(B[i]\) contains the number of elements in the set]
        with represtative \(i\).
    \item Array \(L\) where \(L[i]\) contains poitners to the head and tail of a linked list
        containing elements of the set with represtative \(i\).
\end{enumerate}

\paragraph{Union-Find}
Given sets \(I, J\) with represtatives \(i, j\), the union is as follows:
\begin{enumerate}
    \item Assume \(B[i] \geq B[j]\) without loss of generality.
    \item For each \(m \in J\) update \(A[m] = i\).
    \item Update \(B[i] = B[i] + B[j]\).]]
    \item Append \(L[j]\) to \(L[i]\) and replace \(L[j]\) with an empty list.
\end{enumerate}

Observe that the new \(B[i]\) is atleast twice the old value of \(B[j]\) since
\(B[i] \geq B[j]\). That is, the number of elements in \(B[A[m]]\) changes, it



